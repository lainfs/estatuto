\documentclass[12pt, onecolumn]{article}

\usepackage[utf8]{inputenc}
\usepackage[brazil]{babel}
\usepackage[top=2cm, bottom=2cm, rigth=2cm, left=2cm]{geometry}

\begin{document}
	
	\title{ESTATUTO NORMATIVO E REGULAMENTADOR DA LIGA ACADÊMICA DE 
		INFORMÁTICA EM SAÚDE DO DEPARTAMENTO DE INFORMÁTICA DA
		UNIVERSIDADE FEDERAL DO PARANÁ}

	\maketitle
        
	\section{\begin{center}\textbf{CAPITULO I} \end{center}}
		\begin{center} \textbf{Da Fundação, Sede, Denominação,
                 		Finalidade e Filiação}
		\end{center}
		
                 \textbf{Artigo 1°} - A Liga Acadêmica de Informática em Saúde da
                 Universidade Federal do Paraná (LAINFS - UFPR) é uma entiedade
                 autônoma, civil, não religiosa, política e sem fins lucrativos,
                 vinculada ao Departamento de Informática. Criada em 11 de outubro
                 de 2022 registrada no OFICIO n°10/2022 da Assembleia do
                 Centro Estudantil de Informática Biomédica e foi constituida por
                 tempo indeterminado. Organizada por discentes de Informática
                 Biomédica e orientada por professores do Departamento de Informática
                 da Universidade Federal do Paraná. Possui sede e foro, juntamente
                 com o Departamento de Informática da Universidade Federal do Paraná,
                 na Rua Evaristo F. Ferreira da Costa, 383-391 - Jardim das Américas,
                 Curitiba - PR, sendo regida pelas normas do presente estatuto a partir
                 da data registrada. \\
		
                 \hspace{1cm}\textbf{§ 1°} - A expressão "Liga Acadêmica de
                 Informática em Saúde da Universidade Federal do Paraná"
                 passará a ser designada, daqui por diante, somente pela
                 sigla LAINFS. \\

                 \hspace{1cm}\textbf{§ 2°} - A expressão "Departamento de Informática da
                 Universidade Federal do Paraná" passará a ser designada, daqui por
                 diante, somente pela sigla DInf - UFPR. \\


		\textbf{Artigo 2°} - A LAINFS possui as seguintes finalidades, além
                 de outras que possa eventualmente vir a ter:

                 \hspace{1cm}\textbf{I. ENSINO}: A LAINFS tem o compromisso de fornecer
                 conhecimento teórico-prático, seja mediante o desenvolvimento de
                 atividades internas (seminários, encontros, feiras de desenvolvimento)
                 formuladas pelas diretorias e submetidas ao conselho,
                 seja mediante a atividades externas
                 (desenvolvimento de projetos com uma instituição externa, desenvolvimento
                 de projeto com empresa privada ou pública, palestras e atividades
                 práticas em espaços de ensino público) com temas focados em Informática
                 em Saúde; \\
		
                 \hspace{1cm}\textbf{II. SOCIAL}: A LAINFS tem o compromisso de atuar
                 junto à sociedade em especial as de baixas condições socio-econômicas
                 e de baixas condições socio-educacionais como é o caso das periferias;

                 \hspace{1cm}\textbf{III. CIENTÍFICA}: A LAINFS tem o compromisso de
                 desenvolver ações de cunho científico, como desenvolvimento de projetos,
                 cursos, seminários, além da produção de artigos científicos. \\

                 \textbf{Artigo 3°} - A LAINFS encontra-se aberta a possíveis parcerias;
                 estas serão analisadas pelo responsavel de cada diretoria e submetidas
                 ao conselho. \\
                 
		 \textbf{Artigo 4°} - A LAINFS é submetida ao DInf - UFPR, a qual vai
                 atuar como entiedade regulamentadora da Liga.

        \section{\begin{center}\textbf{CAPITULO II}\end{center}}

\end{document}
